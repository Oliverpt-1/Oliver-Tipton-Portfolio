\documentclass[11pt]{article}
\usepackage{fullpage}
\usepackage{clrscode3e}
\usepackage{amsmath,amsthm,amssymb}
\usepackage{color}
\usepackage[shortlabels]{enumitem}
\usepackage{multicol,multirow}
\usepackage{csquotes}
\usepackage[super]{nth}


\usepackage{tikz}
\usepackage{pgfplots}
\usepgfplotslibrary{ternary, units}
\usetikzlibrary{decorations.pathmorphing, pgfplots.ternary, pgfplots.units}

\setlength{\parskip}{2mm}
\setlength{\parindent}{0mm}

\newcommand{\titlebox}[3]{
    \begin{center}
        \framebox{
            \vbox{
            \hbox to \textwidth { #1 \hfill #3}
            \vspace{-4mm}
            \hbox to \textwidth {\hfill \Large \bf #2 \hfill}
        }
    }
    \end{center}
}

\renewcommand*\arraystretch{1.5}

\newcommand{\answer}[1]{
\vspace{.5\baselineskip} \hrule \vspace{.5\baselineskip}
#1
\vspace{.5\baselineskip} \hrule \vspace{.5\baselineskip}
}

\begin{document}

\titlebox{CSC 383, S'23}
{Homework 3}
{Due Mar. \nth{3}}

\textbf{Directions:}

Write your solutions using Python, Jupyter, and \LaTeX.
Then submit the files \texttt{hw3.tex}, \texttt{hw3.pdf}, and \texttt{2p\_Nash.ipynb}.



\subsection*{Problem 1}

Prove that for any finite game, the set of mixed-strategy profiles is convex.
That is, in a game with $P$ players and $A_p$ actions for player $p$, given any two mixed-strategy profiles $\vec{\sigma}_1, \vec{\sigma}_2 \in \Delta^{A_1} \times \ldots \times \Delta^{A^P}$, and any constant $\alpha \in [0,1]$ the profile $\alpha \vec{\sigma}_1 + (1 - \alpha) \vec{\sigma}_2 \in \Delta^{A_1} \times \ldots \times \Delta^{A^P}$.

\answer{

Since the summation of probabilities in $\sigma_{1} = 1$ and $\sigma{2} = 1$, the profile $\alpha \vec{\sigma}_1 + (1 - \alpha) \vec{\sigma}_2 = 1$ for all players.  We know that this strategy is in the simplex because it sum to 1.  
$ $\newline

\[\triangle_{p} = \alpha \vec{\sigma}_{1}^{p} + (1-\alpha) \vec{\sigma}_{2}^{p} \] This represents the simplex per player.  For each probability in the mixed profile, when multiplied by $\alpha$ and $1-\alpha, $ it will still sum to 1.  As a result, it will still be considered a simplex in the set of possible simplices for that player.
$ $\newline

\[ \prod_{n=1}^{\infty} \triangle_{n} \]
By the definition of a simplitope, the above expression will perform the cartesian product across all simplexes (all players).  The resulting cartesian product will be our simplotope.

}

\subsection*{Problem 2}
In the following Rock-Paper-Scissors variant, compute the Brouwer labels for the vertices of a 3-way subdivision of the symmetric mixed-strategy simplex.

\begin{multicols}{2}
\begin{tikzpicture}
 \begin{ternaryaxis}[
 xmin=0,
 xmax=1,
 ymin=0,
 ymax=1,
 zmin=0,
 zmax=1, 
 xtick={0, .3333, .6666, 1},
 ytick={0, .3333, .6666, 1},
 ztick={0, .3333, .6666, 1},
 xticklabels={3,1,1},yticklabels={1,1,1},zticklabels={2,2,2}
]
\addplot3+[only marks] table {
x      y     z
.0    .0    1.     
.0    1.    .0     
1.    .0    .0     
.0 .3333 .6666
.0 .6666 .3333
.3333 .6666 .0
.3333 .0 .6666
.6666 .3333 0
.6666 0 .3333
.3333 .3333 .3333
};
\
\end{ternaryaxis}
\node [] at (3.45,6.3) {$R$};
\node [] at (-.3,-.1) {$P$};
\node [] at (7.1,-.1) {$S$};
\end{tikzpicture}

\columnbreak

{\Large
\begin{tabular}{ c | c | c | c |}
\multicolumn{1}{c}{} & \multicolumn{1}{c}{$R$} & \multicolumn{1}{c}{$P$} & \multicolumn{1}{c}{$S$} \\ \cline{2-4}
$R$ & $0,0$ & $-1,1$ & $2,-1$ \\ \cline{2-4}
$P$ & $1,-1$ & $0,0$ & $-1,2$ \\ \cline{2-4}
$S$ & $-1,2$ & $1,-1$ & $0,0$ \\ \cline{2-4}
\end{tabular}
}
\end{multicols}


\answer{
^^ The middle node of the simplex is labeled as 1 (can be 1 or 2 so we will break the tie arbitrarily).  
For the given profile of RPS $\begin{bmatrix} 0 \\ 1/3 \\ 2/3 \end{bmatrix}$ the deviation payoffs are: \newline
R $=\frac{1}{3}(-1)+\frac{2}{3}(2) = 1$ \newline \newline
P $=\frac{2}{3}(-1) = -\frac{2}{3}$ \newline \newline
S $=\frac{1}{3}(1) = \frac{1}{3}$ \newline \newline

The utility of the strategy is: \newline
$u_{1} =(\frac{1}{3})(\frac{-2}{3})+\frac{2}{3}(1) = (\frac{4}{9}$) \newline \newline

Therefore the gain is : $1-\frac{4}{9}=\frac{5}{9}$
and the advantage for S is: $\frac{11/9}{14/9} =0.78$  and the advantage for P is: $\frac{3/9}{14/9}=0.22$ Therefore P was most decreased so we label the point 2.


}

\subsection*{Problem 3}

In the Jupyter notebook \texttt{2p\_Nash.ipynb} implement the function to identify Nash equilibria in 2-player games via linear programming.
Note that this version of the notebook contains lots of extra examples; look them over before you get started!



\end{document}