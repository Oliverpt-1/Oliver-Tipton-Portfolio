\documentclass[11pt]{article}
\usepackage{fullpage}
\usepackage{clrscode3e}
\usepackage{amsmath,amsthm,amssymb}
\usepackage{color}
\usepackage[shortlabels]{enumitem}
\usepackage{multicol,multirow}
\usepackage{csquotes}
\usepackage[super]{nth}


\usepackage{tikz}
\usepackage{pgfplots}
\usepgfplotslibrary{ternary, units}
\usetikzlibrary{decorations.pathmorphing, pgfplots.ternary, pgfplots.units}

\setlength{\parskip}{2mm}
\setlength{\parindent}{0mm}

\newcommand{\titlebox}[3]{
    \begin{center}
        \framebox{
            \vbox{
            \hbox to \textwidth { #1 \hfill #3}
            \vspace{-4mm}
            \hbox to \textwidth {\hfill \Large \bf #2 \hfill}
        }
    }
    \end{center}
}

\renewcommand*\arraystretch{1.5}

\newcommand{\answer}[1]{
\vspace{.5\baselineskip} \hrule \vspace{.5\baselineskip}
#1
\vspace{.5\baselineskip} \hrule \vspace{.5\baselineskip}
}

\begin{document}

\titlebox{CSC 383, S'23}
{Homework 5}
{Due Mar. \nth{22}}

\textbf{Directions:}

Write your solutions using Python, Jupyter, and \LaTeX.
Then submit the files \texttt{hw5.tex}, \texttt{hw5.pdf}, \texttt{zero-sum.ipynb}, and \texttt{correlated\_LPs.ipynb}.





\subsection*{Problem 1}


In each of the following cases, describe how many linear programs need to be solved, and for each linear program how many variables and constraints will be needed.
Briefly explain your reasoning.

\begin{enumerate}[(a)]

\item
Finding a Nash equilibrium in a 2-player zero-sum game with $A_1$ actions for player 1 and $A_2$ actions for player 2.

\item
Finding a Nash equilibrium in a 2-player general-sum game with $A_1$ actions for player 1 and $A_2$ actions for player 2.

\item
Finding a correlated equilibrium in a $P$-player game with $A_p$ actions for each player $p \in \{1 \ldots P\}$.

\item
Finding a coarse correlated equilibrium in a $P$-player game with $A_p$ actions for each player $p \in \{1 \ldots P\}$.

\end{enumerate}

\answer{

\begin{enumerate}[(a)]

\item
There are 2 LPs, one has A$_1$ variables and A$_2$ constraints. The other has A$_2$ variables and A$_1$ constraints.
\newline
LP: There is one LP for each player.
\newline
Constraints: Each LP has the other player's constraints because they are trying to minimize that player's outcome through these actions.
\newline
Variables: Each LP has the current player's actions as the variables because they are the payoffs being utilized to minimize the other player's expected utility.

\item
There are $(2^{A_1}-1)(2^{A_2}-1)$ LPs, with $2 + A_1 + A_2$ total constraints and $A_1 + A_2$ total variables.
\newline
LP: For every single LP you have the option to leave in a variable or take it out and you can't have an LP with no variables so you have to delete that case with a minus one.
\newline
Constraints: Each LP has the other player's constraints plus two because they are trying to minimize that player's outcome through these actions and ensuring each player's sum of probabilities are equal to 1.
\newline
Variables: Each LP has both players' actions as the variables because they are the payoffs being utilized to minimize the other player's expected utility.

\item
There are  $\vert p\vert$ LPs, with $\prod_{i=1}^{p} (A_i)$ variables and $\sum_{i=1}^{p} (A_i)(A_i - 1)$ total constraints.
\newline
LP: There is one LP for each player because they are trying to maximize their utility.
\newline
Constraints: There is a constraint for each of the actions that a player can take.
\newline
Variables: There is a variable for each possible outcome therefore you multiply the actions of each player together.  

\item
There are  $\vert p\vert$ LPs, with $\prod_{i=1}^{p} (A_i)$ variables and $\sum_{i=1}^{p} (A_i)$ total constraints.
\newline
LP: There is one LP for each player because they are trying to maximize their utility.
\newline
Constraints: You have to account for each combination of actions along with probability distribution.
\newline
Variables: There is a variable for each possible outcome therefore you multiply the actions of each player together.
\end{enumerate}

}

\subsection*{Problem 2}

In the Jupyter notebook \texttt{zero-sum.ipynb}, implement the function to solve a 2-player zero-sum game.
Note that this version assumes (but does not check) that it was given a zero-sum game.
If you happen to have finished the \texttt{check\_zero\_sum} function from class, feel free to add back the assertion that calls it.




\subsection*{Problem 3}


In the Jupyter notebook \texttt{correlated\_LPs.ipynb}, implement the functions to identify the best utilitarian correlated and coarse correlated equilibria.
Note that there are now two examples of how to construct the CCE linear program.
The first corresponds to the one from class, but includes some key bug-fixes.
The second shows a few examples of techniques you could use for generalizing the LPs for both CE and CCE.



\end{document}