\documentclass[11pt]{article}
\usepackage{fullpage}
\usepackage{clrscode3e}
\usepackage{amsmath,amsthm,amssymb}
\usepackage{color}
\usepackage[shortlabels]{enumitem}
\usepackage{multicol,multirow}
\usepackage{csquotes}
\usepackage[super]{nth}


\usepackage{tikz}
\usepackage{pgfplots}
\usepgfplotslibrary{ternary, units}
\usetikzlibrary{decorations.pathmorphing, pgfplots.ternary, pgfplots.units}

\setlength{\parskip}{2mm}
\setlength{\parindent}{0mm}

\newcommand{\titlebox}[3]{
    \begin{center}
        \framebox{
            \vbox{
            \hbox to \textwidth { #1 \hfill #3}
            \vspace{-4mm}
            \hbox to \textwidth {\hfill \Large \bf #2 \hfill}
        }
    }
    \end{center}
}

\renewcommand*\arraystretch{1.5}

\newcommand\abs[1]{\left|#1\right|}

\newcommand{\answer}[1]{
\vspace{.5\baselineskip} \hrule \vspace{.5\baselineskip}
#1
\vspace{.5\baselineskip} \hrule \vspace{.5\baselineskip}
}

\begin{document}

\titlebox{CSC 383, S'23}
{Homework 8}
{Due Apr. \nth{24}}

\textbf{Directions:}

Write your solutions using and \LaTeX.
Then submit the files \texttt{hw8.tex}, \texttt{hw8.pdf}, \texttt{symmetric\_games.ipynb}, and \texttt{AGGs.ipynb}.





\subsection*{Problem 1}

Consider the following congestion game.
\begin{itemize}
\item Resources: $R = \{1,2,3\}$
\item $A_i$ = $\{(1,2),(1,3),(2,3)\} \; \forall i \in P$ 
\item Congestion Costs: \;\; $c_1(x) = x$ \;\;\;\; $c_2(x) = 2x - 2$ \;\;\;\; $c_3(x) = x^2$
\end{itemize}

\begin{enumerate}[(a)]

\item
Fill in the symmetric data structure for this game when $\abs{P} = 4$.

configs:
\begin{tabular}{|@{\hspace{4mm}}c@{\hspace{4mm}}|@{\hspace{4mm}}c@{\hspace{4mm}}|@{\hspace{4mm}}c@{\hspace{4mm}}|@{\hspace{4mm}}c@{\hspace{4mm}}|@{\hspace{4mm}}c@{\hspace{4mm}}|@{\hspace{4mm}}c@{\hspace{4mm}}|@{\hspace{4mm}}c@{\hspace{4mm}}|@{\hspace{4mm}}c@{\hspace{4mm}}|@{\hspace{4mm}}c@{\hspace{4mm}}|@{\hspace{4mm}}c@{\hspace{4mm}}|}
\hline
3&2&2&1&1&1&0&0&0&0 \\ \hline
0&1&0&2&1&0&3&2&1&0 \\ \hline
0&0&1&0&1&2&0&1&2&3 \\ \hline
\end{tabular}

\item
Identify a pure-strategy Nash equilibrium in this game when $\abs{P} = 4$ and when $\abs{P} = 16$.


\end{enumerate}



\answer{

\begin{enumerate}[(a)]

\item 
payoffs:
\begin{tabular}{|c@{\hspace{4mm}}|c@{\hspace{4mm}}|c@{\hspace{4mm}}|c@{\hspace{4mm}}|c@{\hspace{4mm}}|c@{\hspace{4mm}}|c@{\hspace{4mm}}|c@{\hspace{4mm}}|c@{\hspace{4mm}}|c@{\hspace{4mm}}|c@{\hspace{4mm}}|c@{\hspace{4mm}}|}
\hline
-10&-8&-9&-8&-7&-8&-4&-5&-6&-7 \\ \hline
-5&-8&-7&-13&-12&-11&-20&-19&-18&-17 \\ \hline
-7&-8&-10& -11 & -13 & -15 & -16 & -18 & -20 &-22 \\ \hline
\end{tabular}

\item
The Nash Equilibrium when $\abs{P} = 4$ is [2, 1, 1].  The Nash Equilibrium when $\abs{P} = 16$ is [12, 4, 0].  \\

Work for proving [2,1,1] is Nash of 4 player game:\\

\includegraphics[, width=12
cm]{Photografia 2.png}

Let $a_1$ or $a_2$ or $a_3$ denote the utility by playing that action in the original mixture - this is the number which we are comparing the deviations to.  As can be seen by this work, there is no action change for any player that will cause a positive gain by deviating.  There is one player who will have the same utility, however Nash Equilibrium is always greater-than-or-equal-to, since we expect players to sometimes be indifferent.   Therefore, this is a Nash Equilibrium.

Work for proving [12,4,0] is Nash of 16 player game:\\
\includegraphics[, width=12
cm]{Photografia.png}



Let $a_1$ or $a_2$ denote the utility by playing that action in the original mixture - this is the number which we are comparing the deviations to.  As can be seen by this work, there is no action change for any player that will cause a positive gain by deviating.  Therefore, this is a Nash Equilibrium.
\end{enumerate}

}





\subsection*{Problem 2}

In the Jupyter notebook \texttt{symmetric\_games.ipynb}, implement the functions to compute deviation payoffs and deviation gains for a symmetric mixture in a symmetric game.
Then adapt your Nash local search code to find symmetric mixed-strategy Nash equilibria.



\subsection*{Problem 3}

In the Jupyter notebook \texttt{AGGs.ipynb}, implement the functions to compute deviation payoffs and deviation gains for a role-symmetric mixture in a symmetric game.
Then adapt your Nash local search code to find symmetric mixed-strategy Nash equilibria.

Note that the \texttt{ActionGraphGame} class has several updates relative to the one from class.
In particular, you'll find the new attributes \texttt{self.role\_config\_indices} and \texttt{self.neighbor\_masks} helpful for operating over an action's neighborhood.
The former provides a list of which indices in the configuration belong to each role;
the latter provides a mask for which entries in a profile give probabilities for actions in the neighborhood.
Some examples of how to use them have been provided.

Also note that there are some extra functions to implement, but they should both be easy: \texttt{deviation\_payoffs} and \texttt{deviation\_gains}.
Finally, the interface to the helper functions \texttt{config\_prob} and \texttt{repetitions} has changed.
I think this one will be easier to use now that you have the above-attributes for help with indexing, but if you prefer the old interface, you may revert to it.

\end{document}
